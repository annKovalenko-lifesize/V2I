\section{Introduction}\label{introduction}
For nearly two centuries, petroleum has been a main natural resource used to produce many industrial products such as gasoline, diesel, oil, gas, asphalt, and plastic. Over the years, high demand for petroleum products led to a scarcity of natural resources in easy-access areas and forced companies to reallocate their oil and gas (O\&G) extraction sites to remote offshore (\eg~sea) reservoirs \cite{sof3}. Meanwhile, operations at remote sites are very costly and constrained with limited crew and equipment resources. Moreover, petroleum extraction is a fault-intolerant process that requires ultra-high reliability, specifically in the face of a disaster (\eg~oil spill \cite{FINGAS20149}). Therefore, achieving efficient and safe petroleum extraction - especially when coupled with the location constraints of remote reservoirs is a challenging task for O\&G companies.

\begin{figure}[h!]
	\includegraphics[scale=0.50]{Figures/SmartOilField.pdf}
	\caption{A smart oil field equipped with sensors and the edge node .\label{fig:scen}}
\end{figure}

In order to address these problems, smart oil fields have been proposed and deployed in the industry. Smart oil fields include a diverse set of sensors (\eg~ Gas density, Pipeline pressure, Temperature sensors, Fire / Gas / H2S alarms, Flow monitoring \& Tank levels) and computational facilities. Real-time monitoring of the site, including rigs' structure, wells, distribution lines, etc., is performed to avoid oil and gas leakage, identify corrosion level of the infrastructure, and predict potential future incidents to maximize production efficiency and minimize negative environmental impacts. Therefore, a large amount of raw sensor data is generated every day. With an absence of a full management crew at the site real-time decision making depends on the ability to quickly process and analyze large amounts of data. The situation can become extremely hazardous due to uncertainties in O\&G extraction process (stemming from stochastic gas pressure in the reservoir and leakage of hazardous gases such as H2S) especially in presence of on-site human workers. This imposes a major challenge - data management and analytics. According to the Cisco Public white paper on New Realities in Oil and Gas\cite{Cisco} 48 \% of all respondents involved in O\&G industry admitted that proper data management and analysis is the major challenge for acquiring the best efficiency from the smart oil fields.

During the extraction process, a typical offshore oil field generates between one to two terabytes of raw data per day~\cite{sof2}.  Most of the generated data, such as those pertaining to drilling-platform safety, are time-sensitive and must be processed in real-time to be effective for decision makers. For instance, data obtained from sensors to monitor the release of toxic gases (\eg Hydrogen Sulphide ($H_2S$) which is common in remote oil fields) need to be processed in less than five seconds to preserve workers' safety~\cite{PANDEY201287}. Processing such volume of data requires high-end communication and computation facilities that are not available in remote (offshore) oil fields. Satellite connection is the common vehicle of transmitting data from oil fields to onshore Data Centers. However, the bandwidth of such connections ranges from 64 Kbps to 2 Mbps, making it 12 days to transmit one day's worth of oilfield data to an onshore Data Center~\cite{sof2}. However, establishing a reliable low-latency communication with onshore Data Centres is very complicated for highly distributed offshore oil fields due to high costs and technical difficulty\cite{Cisco}. Therefore, the major challenges for the remote offshore oil fields can be specified as follows: 

\begin{itemize}
\item Real-time processing of resource-intensive emergency applications.
\item Constant decision-making during the extraction process.
\item Real-time monitoring of the site.
\end{itemize}

To address these challenges, there is a need for a \emph{self-organizing cyber-physical system (CPS)} that can collect sufficient information from oil fields, process the data instantly, and make necessary decisions for a seamless and reliable O\&G extraction at remote sites. Realizing such CPS mandates an advanced \emph{integrated communications-computing (cyber) system} that can meet specific data transfer and processing requirements of the oil extraction (\emph{physical}) system in remote oil fields. Despite recent technological advancements, to date, no comprehensive solution exists that can support bandwidth and computationally intensive operations expected in smart oil fields. Most of the existing communications protocols rely on connectivity to a nearby cellular Base Stations which do not exist in remote sites. Moreover, satellite communications do not support enough capacity for a fast wireless connection between control centers and remote oil fields. Provided the difficulties in achieving a real-time response required for disaster management applications in remote smart oil fields, enabling smartness for remote oil fields remains as an open challenge. Edge computing systems, if deployed cleverly, has the potential to obviate these difficulties in remote oil fields and enable them to take advantage of services offered by disaster management applications. In this regard, fast and reliable Wireless Communications and Edge Computing are the main pillars of self-organizing remote smart oil fields.

Our approach to the problems of smart oil fields in remote areas is to use an Edge Computing system in the oil field. Our system is aware of the Quality of Service(QoS) demands of emergency and other applications types in smart oil fields. It accounts for performance characteristics of the computational resource and their limited availability in the edge environment. It also considers low and unreliable connectivity to the onshore Data Centers. In this research, we focus particularly on the emergency application (e.g., those to detect and manage oil spill). Early detection of such disasters can save human lives, minimize environmental impacts, and reduces the cost of disaster recovery. However, these applications are time-sensitive and resource-intensive, hence, require a low-latency and ultra-reliable connectivity to high-end computational resources and to remote monitoring centers. To tackle this challenge our cyber physical system (CPS) integrates novel solutions from wireless communications and Edge Computing to optimize petroleum extraction process at offshore remote oil fields. 

The specific question our research addresses is \emph{how to efficiently process resource-intensive and time-sensitive applications in the presence of a weak and unreliable connection to onshore Data Centers?} Efficiency here refers to solutions that are aware of the connectivity, limited computational capacity, and resource intensiveness of emergency management applications in remote oil fields. The main problem we concentrate on in this paper can be defined as \textit{how to allocate arriving tasks to an Edge Node or in cloud Server in a wireless network so that the number of tasks missing their deadline is minimized?} 

The contributions of this paper are as follows:

\begin{itemize}
    \item Proposing a model that provides seamless wireless communications with high data rate and low latency among sensors, cameras, robots, and user equipment (UE).
    \item Developing a coordinator heuristic that functions based on the proposed model and provides robust task processing for real-time monitoring and decision making.
    \item Implementing an efficient use of limited computational resources to minimize reliance on onshore resources.
    \item We analyze the performance of our proposed heuristic under various workload conditions.
\end{itemize}

The rest of the paper is organized as follows. Section 2 introduces the system model with formulation, assumption and system model scenario. Section 3 discusses allocation approach. Sections 4 and 5 present heuristics and performance evaluation where experimental setup and experiments with the results are particularly described. Section 6 presents the related work. Finally, section 7 concludes the paper.






