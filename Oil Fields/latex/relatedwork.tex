\section{Related work}\label{Related work}
Edge Computing concepts have been previously proposed in the literature for delay tolerant networks. Lorenzo \etal in ~\cite{lorenzo2017robust} proposed resource allocation methods for Edge Computing environments that consider unreliable network connectivity. However, in similar kind of research works \cite{7926320,wang13,zhu17} authors neither consider the case of emergency management applications nor the heterogeneity of the Edge resources, while performing resource allocation.

The more specific problem of resource provisioning for real-time disaster management applications in Edge Computing with low-connectivity to the back-end Data Centers has not been explored in the context of remote smart oil fields and there is a limited research on these issues in other contexts. Efforts towards smart oil fields have been predominantly on analyzing the big data extracted from oil wells by Cameron \etal in~\cite{cameron14} or applying machine learning methods to reduce exploration or drilling costs by Parapuram \etal in ~\cite{mehdi2017}. Warning systems for early prediction of disasters were analyzed by Xu \etal in \cite{xu17}. These solutions are all reliant on onshore Data Centers~\cite{jun15} which are not viable for remote and offshore oil fields.

In other domains, a common solution for low connectivity and allocation of resource-intensive applications is based on the Federation of Edge systems is proposed by Zhang \etal in \cite{zhang2010load} which is not always an option in remote smart oil fields.

To date, limited work exists from academia and industry to design wireless communication networks for remote smart oil fields \cite{sof2,sof1,sof3,sof4,sof5, sof6}. Particularly, prior art cannot address the key challenges of remote operations, due to the following reasons. \emph{First}, the works in \cite{bogaert2004improving} rely on satellite communications between the oil rigs and offshore management centers. However, satellite communication is not suitable for real-time decision making during oil extraction process, as the delay can be substantially large. \emph{Second}, the works in \cite{prabhu2017smart} assume the existence of a macro cell BS at a nearby onshore location that provides wireless support for oil rigs. Nonetheless, remote reservoirs can be very far away from the shore. \emph{Third}, most of existing networks \cite{reza2010applications} operate at sub-6 GHz frequency bands with limited capacity that cannot manage large data rates and URLLC requirements of smart oil fields. \emph{Fourth}, ad-hoc communications protocol with random channel access cannot satisfy the ultra-reliability requirements in smart oil fields with a dense number of wireless devices. \emph{Fifth}, existing works do not provide theoretical foundations for performance analysis and wireless resource management in smart oil rigs.