\begin{abstract}
Mining at remote offshore reservoirs is a common practice in Oil and Gas(O\&G) industry nowadays. Efficient and safe petroleum extraction in such hazardous locations is a challenging task. To address this challenge smart oil fields have been proposed and implemented in remote offshore locations. Smart oil fields include a wide range of sensors to automate the system in order to decrease manpower required at remote locations. These sensors(\eg Gas density, Pipeline pressure, Temperature sensors) generate a huge amount of raw data which needs to be managed and analyzed immediately due to its emergency decision-making applications. The existing solution (satellites and distant cloud servers) isn't feasible due to a significant delay in transfer time. Nevertheless, contingency analysis is a critical activity in the context of the power infrastructure because it provides a guide for resiliency and enables the grid to continue operating even in the case of failure. For this reason, we propose a robust resource allocation model using edge computing concept which is aware of the connectivity, limited computational capacity, and resource intensiveness of the emergency management applications at remote oil fields. To achieve robustness, we propose a resource allocation model that can coordinate tasks (i.e., dynamically utilize available resources), in emergency situations. The proposed model efficiently allocates tasks to appropriate resources to support real-time disaster management applications and therefore, can make smart oil fields safer for people and environment. System evaluation shows that proposed model decreases task deadline miss rate by up to 5\% compare to conventional cloud architecture although edge has much less computational power compared to cloud. 



\end{abstract}

\begin{IEEEkeywords}
Cloud server, Edge Node, smart oil field, cyber physical system.
\end{IEEEkeywords}



% Contingency analysis is a critical activity in the context of the power infrastructure because it provides a guide for resiliency and enables the grid to continue operating even in the case of failure. In this paper, we augment this concept by introducing SOCCA, a cyber-physical security evaluation technique to plan not only for accidental contingencies but also for malicious compromises. SOCCA presents a new unified formalism to model the cyber-physical system including interconnections among cyber and physical components. 
% 
% The cyber-physical contingency ranking technique employed by SOCCA assesses the potential impacts of events. Contingencies are ranked according to their impact as well as attack complexity.  The results are valuable in both cyber and physical domains.  From a physical perspective, SOCCA scores power system contingencies based on cyber network configuration, whereas from a cyber perspective, control network vulnerabilities are ranked according to the underlying power system topology.  
% 
