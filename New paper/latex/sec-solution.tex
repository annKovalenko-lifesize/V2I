\graphicspath{ {Figures/} }
\section{Approach}\label{sec:solution}
A vehicle-generated task arrives at the \bs~through the uplink channel and enters the load balancer. The Load balancer is the component that allocates arriving tasks to appropriate \bs s. It makes the decision how to allocate the task, considering that its robustness must be maximized. Every \bs~has two matrices. One is the Estimated Task Completion (ETC) time matrix \cite{ali2000representing}. Another one is the Estimated Task Transfer (ETT) time matrix. These two matrices help the load balancer to work efficiently.

The ETC Matrix contains estimated task completion time distributions (X$\sim \mathcal{N}(\mu,\,\sigma^{2})$) for different task types in different \bs s. Each ETC matrix cell contains two values, $\mu$ (mean) and $\sigma$ (standard deviation). Together, this two values represent a normal distribution. This distribution is based on historical execution times of different task types on different \bs s. In the ETC matrix, every column defines a \bs s and every row defines a task type.To capture the difference between the estimated completion times of task types we consider the worst-case scenario. Thus, we estimate completion time as the average of the previous completion times summed with its standard deviation. 

The Estimated Task Transfer (ETT) time matrix also stores normal distributions. This distributions represent historic task transfer times for different task types from a receiving \bs~to the neighboring Base Stations. When a task is transferred from one \bs~to another, its transfer time is used to generate a normal distribution (Y$\sim \mathcal{N}(\mu,\,\sigma^{2})$) with respect to its task type. The distributions are saved as an average ($\mu$) and standard deviation ($\sigma$) in ETT matrix. Due to a real-time situation, the two matrices will be updated periodically. Thus, the cloud server sends a periodic pulse which updates the ETC and ETT matrices according to the current status of the system.

When the load balancer gets the task $t_i$  of task type "i", it calculates the probability ($P_i^j$) of this task to meet its deadline $\delta_i$ across the \bs s. For the receiving \bs~"j", the probability can be defined as $P_i^j$($\gamma_i^j$ \textless $\delta_i$ ) = $P_i^j$(Z \textless z) where "z" is ($\delta_i$- $\mu_i^j$) / $\sigma_i^j$. We standardize the distribution with $\mu_i$  = 0 and $\sigma_i$  = 1. After calculating the z value, the probability can be found from the z-score table. For all of the neighboring \bs s, before calculating the probability, we convolve the ETC matrix normal distribution with respective ETT matrix normal distribution. This convolution is important for the correct estimation of the probability of the task to meet its deadline in the neighboring \bs s. This means we need to account for the transfer time as well as for the actual completion time. The convolved distribution is also a normal distribution which can be defined as: W$\sim \mathcal{N}(\mu,\,\sigma^{2})$ = X$\sim \mathcal{N}(\mu,\,\sigma^{2})$$\circledast$Y$\sim \mathcal{N}(\mu,\,\sigma^{2})$\\ 

\begin{figure}[h!]
	\includegraphics[scale=0.35]{ap4.jpg}
	\caption{A proposed model where the load balancer efficiently allocates arriving tasks}
\end{figure}

The resulting distribution (W$\sim \mathcal{N}(\mu,\,\sigma^{2})$) is used to calculate the probability of the task in a specific \bs. If a neighboring \bs~is "k" and task type is "i" then the probability can be defined as "$P_i^k$" where z =  ($\delta_i$ - $\mu_i^k$) / $\sigma_i^k$. When the probability of the received task in all of the \bs s (receiving and neighboring) is calculated, it is allocated to the \bs~that offers the highest probability for the task to be completed within the deadline. When the task'��s probability to meet its deadline is zero (0), the task is dropped. Therefore this task will not be allocated to any of the \bs s, nor it will enter the batch queue of any \bs~and increase the historic mean of completion times. Task dropping procedure during the over subscription situation will implicitly increase the probability of the other tasks to meet their deadlines.