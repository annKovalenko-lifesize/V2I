
\section{Introduction}\label{sec:intro}
Recent advancements in communication and computation technologies have stimulated a rapid development of vehicular networks. Federal Communications Commission (FCC)~\cite{ali2011co} has reserved 5.850 to 5.925 GHz frequency band for Vehicle-to-Everything (\emph{V2X}) communications. Vehicle-to-Infrastructure (V2I) communication is one prominent form of V2X that draws the majority of work to itself. In V2I, infrastructure refers to all edge and core technologies that facilitate communications and computations for vehicular requests.

As shown in Figure~\ref{fig:scen}, autonomous vehicles send their service requests (tasks) to \bs s while operating on the road. A \bs~(BS) is capable of communicating with vehicles and processing vehicular tasks~\cite{bok2016multiple}. Upon the completion of the task processing its results are sent back to the requesting vehicle. Examples of such vehicular tasks can be a Wrong Way Driver warning~\cite{ABIresearch}, Cooperative Forward Collision warning~\cite{ElBatt}, and Lane Change warning~\cite{ABIresearch}. This type of tasks can only tolerate a short end-to-end delay~\cite{ali2011co}. For such delay sensitive tasks, there is no value in executing them after a tolerable delay. In this research, we model an acceptable end-to-end delay as an individual hard deadline for the task. 

\begin{figure}[h!]
\includegraphics[scale=0.38]{Figures/v2i_scenario.pdf}
\caption{A Vehicle to Infrastructure (V2I) scenario where vehicles send requests to a Base Station and receive the response. A Base Station is a roadside unit with communication and computation abilities.\label{fig:scen}}
\end{figure}

Related literature suggests that in order to enhance the performance, V2I systems should incorporate wireless network infrastructure and cloud computing capabilities \cite{yu2016optimal}. For instance, the Vehicular Cloud Computing (VCC) was initiated in \cite{gerla2012} as a subsection of Mobile Cloud Computing. However, both VCC and conventional cloud systems incur high latency~\cite{li2017resource}, thus, cannot be used for delay intolerant tasks. Nonetheless, \bs' computational power can be harnessed with an edge computing system. With the edge device, vehicular services can be managed directly at the \bs with low latency and with no necessity to communicate with the cloud. Various \bs s in a V2I system can potentially be heterogeneous, both in terms of computational characteristics and communication medium to the core network (\eg wireless, optical, and wired \cite{bok2016multiple}). 

Significant problems arise during road emergencies (\eg road accidents and disasters) when a rapid increase in service requests to \bs s significantly affects the tasks' service time. In fact, in this situation, \bs resources become oversubscribed, and it cannot provide enough computational power for all the arriving tasks to meet their deadlines. Accordingly, our goal, in this research, is to design the V2I system to be robust against uncertain task arrival. In the literature, \emph{Robustness} is defined as the degree to which a system can maintain a certain level of performance even with given uncertainties \cite{ali2004measuring,smith2009robust,canon2010evaluation}. In our research, we evaluate robustness of the V2I system according to the number of tasks that can meet their deadlines. The main question we try to answer is how to allocate arriving tasks among the \bs s so that the system stays robust? Or, in other words, we try to find a way to maximize the number of tasks meeting their deadlines. Any possible solution to this problem needs to overcome the uncertainties of the system. In particular, uncertainty imposed by indeterministic task arrival rate and by the communication delay. 

Previous research works either discard these uncertainties\cite{bok2016multiple} or focus on the uncertainty introduced by communication\cite{ali2011co}. Alternatively, to assure robustness of the V2I system, we propose a probabilistic resource allocation model that copes with uncertainties introduced by both communication and computation. Our proposed model is aware of the connectivity amongst \bs s (\ie edge nodes) and their heterogeneity. In the face of oversubscription, we devise a Load Balancer at the \bs~level that can leverage the computational capabilities of other \bs s to improve robustness of the V2I system.

Allocating vehicular tasks in a V2I system is proven to be an NP-complete problem~\cite{Ullman1975} \cite{Lenstra1981}. Therefore, a large body of any research on this topic has to been dedicated to developing resource allocation heuristics for V2I systems~\cite{ali2004measuring,yu2016optimal,khemka2014utility,Lenstra1981,li2017resource,pyun2016,liu2010rsu}. We leverage our proposed probabilistic model and establish a novel allocation heuristic for the \bs 's Load Balancer. To evaluate the performance of our proposed heuristic, we simulate the V2I environment and analyze its behavior.

The main contributions of this paper are as follows:

\begin{itemize}
    \item Proposing a model that encompasses the uncertainties exist in communication and computation. The model is leveraged to determine the probability of completing an arriving task within the deadline on different \bs s.
    \item Developing a load balancing heuristic that functions based on the proposed model and increases the robustness of the V2I system.
    \item Analyze the performance of our proposed heuristic under various workload conditions and in comparison to different existing models to show the advantages of the proposed model.
\end{itemize}

The rest of the paper organized as follows. Section 2 describes previous scientific works related to the matter of our research. Section 3 introduces the scenario and assumptions made to describe the model. Section 4 includes the problem statement and the proposed system model with the formulation. Section 5 discusses probabilistic allocation approach. Sections 6 and 7 present heuristics and simulation respectively.  Experimental stage and performance evaluations described in section 8. Finally, section 9 concludes the paper.

