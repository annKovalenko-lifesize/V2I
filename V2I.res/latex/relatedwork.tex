\section{Related Work}\label{relatedwork}
Efficient resource allocation which can decrease the deadline missing rate is the major challenge for V2I systems. Oversubscription situations only make this challenge even more complex. The robustness and QoS can deteriorate due to the lack of efficient resource allocation in a V2I system. Thus, a proficient task allocation decision can decrease latency and significantly improve robustness.  

%decreasing the deadline missing rate of arriving tasks and increasing the robustness (that leads to the reliability) of the system. 

There are several approaches proposed in the literature. In \cite{li2017resource} [Jun Li et al. 2017] propose a local fog resource management model in Fog Enhanced Radio Access Network (FeRANs) based on V2X environment. The core concept of this paper is to improve the QoS at each individual fog node(e.g., \bs~) for real-time vehicular services. Authors propose two resource management schemes. Both schemes prioritize real-time vehicular services over other services. The model considers service migration from one fog node to another based on reserved resource availability.

%to remove unpromising service requests from the waiting queue before scheduling as well as balancing upload and download requests.

[Ali et al. 2011] in \cite{ali2011co} propose a multiple RSU scheduling to provide cooperative data access to multiple RSUs in vehicular ad-hoc networks. They categorize requests or tasks into two types (delay sensitive and delay tolerant) according to their data size.  In oversubscription situation, authors propose to transfer delay tolerant requests to the neighboring RSUs with a lower workload.    
 
In \cite{liu2010rsu}  [Liu and Lee 2010] suggest an RSU-based data dissemination framework to address challenges in vehicular networks. The system aims to efficiently utilize available bandwidth for both the safety-critical and the non-safety-critical services. An analytical model is proposed to investigate the system performance in terms of providing data services with delay constraints.

[Tong et al. 2016] in \cite{tong2016hierarchical} offer a hierarchical architecture of the Edge to maximize the volume of mobile workload served in an oversubscription situation. Authors emphasize the advantage of hierarchical cloud architecture over a flat architecture. Nevertheless, hierarchical architecture also has some problems. For instance, it enables aggregation of the peak loads across different tiers to maximize the workload volume served, which is not efficient. As a solution authors propose a workload placement algorithm for mobile program/workload placement on edge cloud and provisioning computational capacity. 

%transferring delay tolerant requests to the neighboring Base Stations when receiving base station is overloaded. It also proposes a scheduling algorithm which prioritizes delay-sensitive, small, popular tasks over the delay tolerant, large, unpopular tasks.
  
[Adachi et al. 2016] in \cite{adachi2016cloud} suggest a special model for local resource reservation and reallocation based on the priority of services.

Although various extensive research works have been embraced in the field of vehicular systems \cite{maeshima2007, korkmaz2006, mak2005}, they are limited to the issues of communication functionality and quality requirements from networking perspective. However, none of them considered V2I system in perspective of computational capacity and task processing in a \bs~(\eg edge device). Whereas our work can be distinguished as it contemplates a V2I system in both computational and communicational perspectives.