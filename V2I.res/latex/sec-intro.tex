
\section{Introduction}\label{sec:intro}
Recent advancements in communication and computation technologies have stimulated a rapid development of vehicular networks. Federal Communications Commission (FCC)~\cite{ali2011co} has already reserved 5.850 to 5.925 GHz frequency band for Vehicle-to-Everything (\emph{V2X}) communications. Vehicle-to-Infrastructure (V2I) communication is one prominent form of V2X that draws the majority of work to itself. In V2I, infrastructure refers to all edge and core technologies that facilitate communications and computation of vehicular tasks.

As shown in Figure~\ref{fig:scen}, autonomous vehicles send their service requests to \bs s while travelling a road. A \bs~(BS) is capable of communicating with vehicles, and can process vehicular tasks~\cite{bok2016multiple}. Upon task completion on the Base Stations, its results are sent back to the requesting vehicle. Examples of such vehicular tasks can be wrong way driver warning~\cite{ABIresearch}, cooperative forward collision warning~\cite{ElBatt}, and lane change warning~\cite{ABIresearch}. This type of tasks can tolerate a short end-to-end delay~\cite{ali2011co}. End-to-end delay is composed of three contributing factors \cite{mostafa2011}, namely up-link delay, processing delay, and down-link delay. In end-to-end delay ``uplink delay" and ``down-link delay" can be defined as communication delay. Whereas processing delay can be defined as ``computational delay". In case of delay sensitive tasks, there is no value in executing tasks after their tolerable delay. In this research, we model the tolerable end-to-end delay of each task as an individual hard deadline for the task. 

\begin{figure}[h!]
\includegraphics[scale=0.38]{Figures/v2i_scenario.pdf}
\caption{A Vehicle to Infrastructure (V2I) scenario where vehicles send requests to a Base Station and receive the response. A Base Station is a roadside unit with communication and computation abilities.\label{fig:scen}}
\end{figure}

For providing effective support to the vehicular services, it is suggested by the literature that V2I systems should be enhanced by incorporating wireless network infrastructure and cloud computing \cite{yu2016optimal}. For instance, the vehicular cloud computing (VCC) has been initiated in \cite{gerla2012} as a subsection of mobile cloud computing. However, both VCC and conventional cloud systems incur a high latency~\cite{li2017resource}, thus, cannot be used for delay intolerant tasks of V2I systems. Nonetheless, Base Stations' computational power can be harnessed as an edge computing system. With edge computing, vehicular services can be managed directly at the BS with low latency, without requiring to communicate with the cloud. Various \bs s in a V2I system can potentially be heterogeneous, both in terms of computational characteristics and communication medium to the core network (\eg wireless, optical fibre and wired \cite{bok2016multiple}). 

A problem arises during an emergency (\eg a natural disaster and road accidents) when a rapid increase in service requests to \bs s significantly affects the tasks' service time. In fact, in this situation, BS resources become oversubscribed and the BS cannot meet the deadline of all arriving tasks. Accordingly, our goal, in this research, is to design the V2I system to be robust against uncertain task arrival. \emph{Robustness} in the literature is defined as the degree to which a system can maintain a certain level of performance even with given uncertainties \cite{ali2004measuring,smith2009robust,canon2010evaluation}. In this research, we measure robustness of the V2I system as the number of tasks that can meet their deadlines. Therefore, the problem in this case is how to allocate arriving tasks to \bs s, so that the system is robust (\ie the number of tasks meeting their deadlines is maximized)? To achieve robustness, any solution needs to overcome uncertainties of the system. In particular, uncertainty exists in the task arrival to the \bs~and in the communication delay. 

%in their approaches and try to maximize robustness by prioritizing allocation of delay sensitive tasks over other types of tasks. 
Previous research works either discard these uncertainties\cite{bok2016multiple} or focus on the uncertainty introduced by communication\cite{ali2011co}. Alternatively, to assure robustness of the V2I system, we propose a probabilistic resource allocation model that copes with uncertainties introduced by both communication and computation. Our proposed model is aware of the connectivity amongst \bs s (\ie edge nodes) and their heterogeneity. In the face of oversubscription, we devise a load balancer at the \bs~level that can leverage the computational capabilities of other \bs s to improve robustness of the V2I system. %Our model uses historical data to calculate probabilities of certain task types to meet the deadline in respective Base Stations and allocates according to the calculated probabilities.

Allocating vehicular tasks in a V2I system is proven to be an NP-complete problem~\cite{Ullman1975} \cite{Lenstra1981}. Therefore, a large body of research has been dedicated to develop resource allocation heuristics in V2I systems~\cite{ali2004measuring,yu2016optimal,khemka2014utility,Lenstra1981,li2017resource,pyun2016,liu2010rsu}. We leverage our proposed probabilistic model and develop a heuristic for the \bs~load balancer. To evaluate the performance of our proposed heuristic, we simulate the V2I environment and analyze its behavior under various workload conditions.

The main contributions of this paper are as follows:

\begin{itemize}
	\item Proposing a model that encompasses the uncertainties exist in communication and computation. The model is leveraged to determine the probability of completing an arriving task before its deadline on different \bs s.
	\item Developing a load balancing heuristic that functions based on the proposed model and increases the robustness of the V2I system.
	\item We analyze the performance of our proposed heuristic under various workload conditions.
\end{itemize}

The rest of the paper is organized as follows. Section 2 introduces the scenario and assumptions made to describe the model. Section 3 includes the problem statement and the proposed system model with formulation. Section 4 discusses probabilistic allocation approach. Sections 5 and 6 present heuristics and simulation. Experimental stage and performance evaluations are particularly described in section 7. Finally, section 8 concludes the paper.

%A major challenge in V2I systems is the lack of an efficient resource allocation which can decrease the deadline missing rate and increase the robustness (that leads to the reliability) of the system. There are several approaches that have been proposed. \cite{li2017resource} proposes to remove unpromising service requests from the waiting queue before scheduling as well as balancing upload and download requests. In \cite{liu2010rsu} suggests transferring delay tolerant requests to the neighboring Base Stations when receiving base station is overloaded. It also proposes a scheduling algorithm which prioritizes delay-sensitive, small, popular tasks over the delay tolerant, large, unpopular tasks. \cite{adachi2016cloud} suggests special models of local resource reservation and reallocation based on the priority of services.

%The United States Department of Transportation (DoT) considers V2I communications the future of Intelligent Transportation Systems\cite{ali2011co}.
% Being close to end-users, the edge mitigates the delay produced by communication with the cloud.
%The specific research question we addresses in this paper is how to make the V2I system robust? 