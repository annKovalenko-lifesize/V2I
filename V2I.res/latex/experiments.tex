\section{Experiments}\label{experiments}
To study the performance of our model thoroughly, we evaluate the system under various amounts of arriving tasks. We analyze the impact of the increasing workload on the oversubscribed system. Initially, each \bs~in the system has it's own workload and is proved to be oversubscribed (70 - 80\% of arriving tasks miss the deadline). Additionally, we assign a testing workload to the \bs~which uses a load balancer to allocate them. The individual workload for each \bs is independent of the testing workload, as we assume the testing workload to be extra request bursts that can occur during the rush hours or the emergency situations. We consider the minimum testing workload to be a 100 tasks and the maximum to be a 1000 tasks. Each step we increase the testing workload by a 100 tasks and account for the number of tasks that missed their deadline. For each step, we perform 30 experiments and calculate the average number of tasks that miss the deadline. We compare the average number of tasks that missed the deadline while using MR and MECT heuristics. Figure 4 demonstrates the results of described experiments. The first image of figure 4 represents the results obtained using the FCFS scheduling algorithm. As we can see, the load balancer performs significantly better using the MR heuristic rather then MECT. About only 5\% of all tasks of the testing workload miss their deadlines using the MR heuristic, where more than 90\% of the testing tasks miss their deadline with the MECT heuristic. Apparently, the number of additionally arriving tasks does not significantly affect the performance of the algorithms, as the miss rate does not change much with the increase in the number of tasks. We assume that the performance of the MR heuristic depends on the level of the initial oversubscription of the system, but we leave finding the proof for our assumption for the future research. We can also notice that MR heuristic performs slightly better using the FCFS scheduling rather than SJF. The reason for that can be the common nature of all the arriving tasks (the burst times for different times are not much different). Overall, we can observe that MR heuristic performs noticeably better than our baseline heuristic, which can provide a better robustness for the whole system in the situations of oversubscription. 
%\paragraph{\textbf{Impact of Deadline looseness}}
%In this experiment we investigate the impact of deadline looseness on implemented heuristics. For this purpose initial over subscription is created with 1000 tasks workload in every \bs s. In every trial on top of initial over subscription, a batch of 1400 tasks with different arrival time is provided to the system. Slack or ``system slack" is one of the parameter for calculation of deadline. Therefor it is used for generating deadline looseness. In this experiment, for the first trial initial slacks for 3 \bs s are used. After the first trail slacks are increased by adding 10 units to every slack. The reason for adding 10 is that this increased amount has significant impact on deadline miss rate for the heuristics. For scheduling of the arriving tasks again 2 different scheduling policies are used to evaluate proposed system.

%rom figure 4 it is observed that MR Task Dropping heuristic performs better than MR LB and NO LB in terms of deadline missed tasks. The decreasing trends of the heuristic bars reflects the impact of deadline looseness of tasks. One of the interesting observation from this decreasing trend is that the system can work also support delay tolerant tasks with certain deadline looseness. 

%\begin{figure*}[h!]
%\includegraphics[scale=0.6]{DeadLineLoosen_FCFS.pdf}
%\caption{Impact of deadline looseness on heuristics}
%\end{figure*}

%\paragraph{Task Dropping with given probability}
%In this experiment we have used our heuristic with certain threshold value as highest probability for task allocation decision. As such the heuristic allocates those tasks to specific \bs s greater than the threshold probability.
